%narms.tex, an example driver file for Balkema documents.

%use the following for A4 paper:
\documentclass[12pt,a4paper,twocolumn,fleqn]{narms}

% packages needed
\usepackage{subfigure}
\usepackage{epsfig}
\usepackage{timesmt}

% add here more packages based on the document format


% setting math equation indent from left 0pts

\mathindent=0pt%

% use this for chicaco style reference
% Author references
% IMPORTANT: Author wants to format references in chicaco style Author must use BiBTex
% IMPORTANT: Author wants to format numbered references remove chicaco style file and \bibliographystyle{chicaco}

\usepackage{chicaco}

%  \cite{key}
%    which produces citations with full author list and year.
%    eg. (Brown 1978; Jarke, Turner, Stohl, et al. 1985)

%  \citeNP{key}
%    which produces citations with full author list and year, but without
%    enclosing parentheses:
%    eg. Brown 1978; Jarke, Turner & Stohl 1985

%  \citeA{key}
%    which produces citations with only the full author list.
%    eg. (Brown; Jarke, Turner & Stohl)

%  \citeANP{key}
%    which produces citations with only the full author list, without
%    parentheses eg. Brown; Jarke, Turner & Stohl

%  \citeN{key}
%    which produces citations with the full author list and year, but
%    can be used as nouns in a sentence; no parentheses appear around
%    the author names, but only around the year.
%      eg. Shneiderman (1978) states that......
%    \citeN should only be used for a single citation.

%  \shortcite{key}
%    which produces citations with abbreviated author list and year.

%  \shortciteNP{key}
%    which produces citations with abbreviated author list and year.

%  \shortciteA{key}
%    which produces only the abbreviated author list.

%  \shortciteANP{key}
%    which produces only the abbreviated author list.

%  \shortciteN{key}
%    which produces the abbreviated author list and year, with only the
%    year in parentheses. Use with only one citation.

%  \citeyear{key}
%    which produces the year information only, within parentheses.

%  \citeyearNP{key}
%    which produces the year information only.


%%%%%%%%%%%%%%%%%%%%%%%%%%%%%%%%%%%%%%%%%%%%%%%%%%%%%%%%%%%%%%%%%%%
%%%  All this stuff is from modifying the article.cls for Balkema
%%%  specifications.

%\title{...}
%\author{...}
%use \aff for author affiliations
% use \authornext for from second author
% empty line space between multiple authors
%\abstract{...}
%\maketitle{}

%%%%%%% Style for TABLES
% insert tabular command inside \tabletext{} this will produce tables in 10pts


\begin{document}
\title{Effect of initial volume fraction on the collapse of granular columns in fluid}
\author{{K. Kumar} \\
{\aff{Computational Geomechanics Research Group, Department of Engineering, University of Cambridge, UK}} \\
\\
{\authornext{J-Y. Delenne}}\\
{\aff{IATE, UMR 1208 INRA-CIRAD-Montpellier Supagro-UM2, University of Montpellier 2, France.}} \\
\\
{\authornext{K. Soga}}\\
{\aff{Department of Civil and Environmental Engineering, University of California, Berkeley, USA.}}}

\date{}% No date.

\abstract{This paper investigates the effect of initial volume fraction on the runout characteristics of granular column collapse in a fluid. Two-dimensional sub-grain scale numerical simulations are performed to understand the flow dynamics of granular collapse in a fluid. The Discrete Element (DEM) technique is coupled with the Lattice Boltzmann Method (LBM), for fluid-grain interactions, to understand the evolution of submerged granular flows. The fluid phase is simulated using Multiple-Relaxation-Time LBM (LBM-MRT) for numerical stability. In order to simulate interconnected pore space in 2D, a reduction in the radius of the grains (hydrodynamic radius) is assumed during LBM computations. A parametric analysis is performed to assess the influence of the granular characteristics (initial packing) on the evolution of flow and run-out distances. The volume of the initial packing is changed to simulate different stress conditions while maintaining the same aspect ratio. The influence of the stress condition on the run-out behaviour is studied for different permeabilities. The granular flow dynamics is investigated by analysing the effect of hydroplaning, water entrainment and viscous drag on the granular mass. The mechanism of energy dissipation, the shape of the flow front, water entrainment and evolution of packing density is used to explain the difference in the flow characteristics of loose and dense granular column collapse in a fluid.}

\maketitle

\section{INTRODUCTION}

\subsection{Type area}


%\section{FORMULATION OF THE PROBLEM}




\bibliographystyle{chicaco}
\bibliography{references}

\end{document}
